% \documentclass{beamer}

\input{aula}

\usepackage{alltt}
\usepackage{tikz}
\usetikzlibrary{calc}
\usetikzlibrary{shapes.callouts}
\usetikzlibrary{decorations.text}%workaround for undefined \pgf@test

\newcommand{\refnode}[2]{%
  \tikz[remember picture,baseline=-.5ex]{%
    \node[draw](#1){#2};
  }%
}

\newcommand{\pos}[1]{%
  \tikz[remember picture,overlay,baseline=-.5ex]{%
    \node(#1){\phantom{X}};
  }%
}

\PYstyledefault

\begin{document}

\title[PC1-09 Funções]{
  {\large Programação de Computadores 1}\\[2mm]
  \includegraphics[height=2cm]{images/scilab.png}\\
  {\normalsize Capítulo 9}\\
  Funções
}
\subject{Programação de Computadores}
\author{José Romildo Malaquias}
\institute[UFOP]{
  Departamento de Computação\\
  Universidade Federal de Ouro Preto
}
\date{\semester}

\frame{\titlepage}

\frame{\tableofcontents}

\section{Funções}

\begin{frame}[fragile]{Propósitos do uso de funções}
  \begin{itemize}
    \item Modularizar um programa em partes menores
    \item Executar uma tarefa que é frequentemente solicitada
    \item Aumentar a legibilidade e manutenibilidade do programa
    \item Implementar as chamadas UDF (User Defined Functions), para
    complementar as necessidades do programador na execução de tarefas
    não suportadas pelo ambiente de programação.
  \end{itemize}
\end{frame}

\begin{frame}[fragile,allowframebreaks]{Exemplo: maior de dois valores}
  Escrever um programa que
  \begin{itemize}
    \item leia dois valores numéricos,
    \item calcula o maior valor desses dois números, e
    \item mostra o resultado.
  \end{itemize}

  \begin{pygmented}[lang=scilab]
x1 = input("Primeiro valor .....: ");
x2 = input("Segundo valor ......: ");
maior = maior2valores(x1, x2);
printf("O maior valor é: %g\n", maior);
  \end{pygmented}

  \begin{pygmented}[lang=scilab]
clear; clc;

// Definição da função
function resposta = maior2valores(a, b)
    if a > b then
        resposta = a;
    else
        resposta = b;
    end
endfunction

// Programa principal
x1 = input("Primeiro valor .....: ");
x2 = input("Segundo valor ......: ");
maior = maior2valores(x1, x2);
printf("O maior valor é: %g\n", maior);
  \end{pygmented}

  \framebreak
  \mbox{}\vspace{1cm}
  \begin{mdframed}
    \begin{alltt}

\refnode{k1}{\PY{k}{function}} \refnode{r}{resposta} = \refnode{n}{\PY{n+nf}{maior2valores}}(\refnode{a}{a}, \refnode{b}{b})

    \pos{ctl}\PY{k}{if} a > b \PY{k}{then}
        resposta = a;
    \PY{k}{else}
        resposta = b;\pos{cr}
    \pos{cbl}\PY{k}{end}

\refnode{k2}{\PY{k}{endfunction}}
    \end{alltt}    
  \end{mdframed}

  \begin{tikzpicture}[remember picture,overlay,note/.style={rectangle callout, fill=#1}]
    \draw (ctl.north west) -|(cr.north)--(cr.south)|-(cbl.south west)--(ctl.north west);
    \node[note=green!30,align=left,overlay,callout absolute pointer=(r.north)] at ($(r.north)+(-.3,2)$) {parâmetro de saída:\\calculado pela função};
    \node[note=yellow!30,overlay,callout absolute pointer=(b.north)] at ($(a.north)+(2,1)$) {parâmetro de entrada};
    \node[note=yellow!50,align=left,overlay,callout absolute pointer=(a.north)] at ($(a.north)+(2,1)$) {parâmetro de entrada:\\fornecido na chamada da funçâo};
    \node[note=blue!30,overlay,callout absolute pointer=(n.north)] at ($(n.north)+(-.1,1)$) {nome da função};
    \node[note=red!30,overlay,callout absolute pointer=(cr)] at ($(cr)+(2,0)$) {corpo da função};
    \node[note=black!30,overlay,callout absolute pointer=(k1.north)] at ($(k1.north)+(0,1)$) {palavra chave};
    \node[note=black!30,overlay,callout absolute pointer=(k2.south)] at ($(k2.south)-(0,1)$) {palavra chave};
  \end{tikzpicture}
\end{frame}

% \begin{frame}[fragile,allowframebreaks]{Exemplo: calculando com vetores}
%   Escrever um programa que
%   \begin{itemize}
%     \item define três vetores com quatro elementos numéricos, e
%     \item encontra e exibe:
%     \begin{itemize}
%       \item o menor elemento de cada vetor, e
%       \item a média dos elementos de cada vetor
%     \end{itemize}
%   \end{itemize}
%   \framebreak
%   \begin{pygmented}[lang=scilab]
% clear; clc;

% // Função para encontrar o menor elemento de um vetor
% function x = menor(v)
%     n = length(v);
%     x = v(1);
%     for i = 2:n
%         if v(i) < x then
%             x = v(i);
%         end
%    end
% endfunction
%   \end{pygmented}
%   \framebreak
%   \begin{pygmented}[lang=scilab]
% // Função para calcular a média dos elementos de um vetor
% function x = media(v)
%     n = length(v);
%     x = 0;
%     for i = 1:n
%         x = x + v(i);
%     end
%     x = x / n;
% endfunction
%   \end{pygmented}
%   \framebreak
%   \begin{pygmented}[lang=scilab]
% // Programa principal
% vet1 = [ 12, 22, 78, 66];
% vet2 = [ 11, 88, 77, 99];
% vet3 = [ 5, 7, 2, 1];
% printf("vet1: menor: %g, média: %g\n", menor(vet1), media(vet1));
% printf("vet2: menor: %g, média: %g\n", menor(vet2), media(vet3));
% printf("vet3: menor: %g, média: %g\n", menor(vet3), media(vet3));
%   \end{pygmented}
% \end{frame}

\begin{frame}[fragile,allowframebreaks]{Exemplo: número de combinações}
  Escrever um programa que calcula o número de combinações de $n$
  elementos tomados $k$ a $k$, sendo $n$ e $k$ fornecidos pelo usuário.

  \[
  C^n_k = \frac{n!}{(n-k)! \, k!}
  \]

  \framebreak
  \begin{pygmented}[lang=scilab]
clear; clc;

// Função para encontrar o fatorial de um número inteiro
function fat = fatorial(n)
    fat = 1;
    for i = 1:n
        fat = fat * i;
   end
endfunction

// Função para calcular o número de combinações
function x = numCombinacoes(n, k)
    x = fatorial(n) / (fatorial(n - k) * fatorial(k));
endfunction

// Programa principal
n = input("n: ");
k = input("k: ");
nComb = numCombinacoes(n, k);
printf("Número de combinações: %g\n", nComb);
  \end{pygmented}
\end{frame}

\end{document}
