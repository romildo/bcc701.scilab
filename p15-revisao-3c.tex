\documentclass[11pt]{practice}

% Comente a linha seguinte para incluir as soluções
\tcbset{lowerbox=ignored}

\begin{document}

\institution{UFOP\quad DECOM}
\course{Programação de Computadores I}
\subtitle{Aula prática 15}
\title{Revisão 3c}
\author{}
\date{2013--2}
\maketitle

\begin{abstract}
  Esta prática é uma preparação para a terceira prova. Vamos lidar com
  problemas cuja solução envolve o uso des conceitos vistos até agora,
  destacando: vetores, matrizes e funções.
\end{abstract}

%\tableofcontents

\section{Exercícios}

\begin{task}[breakable]{Correção de prova\footnote{Questão 3 da prova 3 de 2013/1}}{}

  Construa um programa na linguagem Scilab para corrigir cinco provas de
  múltipla escolha. Cada prova é composta por dez questões valendo um
  ponto cada. A correção é feita comparando as respostas com um
  gabarito. A resposta de cada questão pertence ao intervalo inteiro [1:
  4]. O gabarito está armazenado em um vetor linha e as provas estão
  armazenadas em uma matriz $5 \times 10$, sendo que o índice da linha
  da matriz corresponde ao número do candidato, conforme os comandos
  abaixo:

  \begin{minted}{scilab}
gabarito = [ 1, 4, 2, 3, 3, 4, 1, 1, 3, 2 ];

provas  = [ 1, 2, 2, 3, 3, 4, 4, 1, 3, 2;   // prova 1
            1, 2, 2, 3, 3, 4, 4, 1, 3, 2;   // prova 2
            2, 4, 2, 3, 3, 1, 4, 1, 3, 1;   // prova 3
            1, 4, 2, 3, 3, 4, 1, 1, 3, 2;   // prova 4
            1, 1, 1, 3, 1, 1, 1, 1, 3, 2 ]; // prova 5
  \end{minted}

  Os dados (gabarito e respostas das provas) não precisam ser digitados
  pelo usuário. Podem ser atribuídos diretamente no programa como
  mostrado acima.

  O seu programa deve:
  \begin{enumerate}
    \item calcular e exibir as notas dos candidatos (não é necessário
    armazenar em um vetor), e
    \item calcular e exibir a média das notas.
  \end{enumerate}

  \begin{runexample}
RESULTADO DA PROVA

Nota do candidato nº 1: 8
Nota do candidato nº 2: 8
Nota do candidato nº 3: 6
Nota do candidato nº 4: 10
Nota do candidato nº 5: 6

Média das notas: 7.6
  \end{runexample}

  \tcblower
  \solution
  \lstinput{scilab}{listings/p15/correcao.sce}
\end{task}

\end{document}
