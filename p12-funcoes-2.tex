\documentclass[11pt]{practice}

% Comente a linha seguinte para incluir as soluções
\tcbset{lowerbox=ignored}

\begin{document}

\institution{UFOP\quad DECOM}
\course{Programação de Computadores I}
\subtitle{Aula prática 12}
\title{Funções: Segunda Parte}
\author{}
\date{2013--2}
\maketitle

\begin{abstract}
  As atividades propostas nesta prática visam aprimorar as noções sobre
  funções definidas pelo própio programador para o desenvolvimento de
  aplicações.
\end{abstract}

%\tableofcontents

\section{Exercícios}

\begin{task}[breakable]{Triângulos}{}

  Crie um programa que receba três valores (obrigatoriamente maiores que
  zero), representando as mediadas dos três lados de um triângulo,
  verifica se estes lados formam um triângulo e, em caso afirmativo,
  classifica o triângulo quanto aos lados.

  Defina funções para:
  \begin{itemize}
    \item determinar se três valores dados podem ser medidas dos lados
    de um triângulo (sabe-se que, para ser um triângulo, a medida de um
    lado qualquer deve ser inferior ou igual à soma das medidas dos
    outros dois), e
    \item classificar o triângulo quanto aos lados (equilátero,
    isósceles ou escaleno).
  \end{itemize}

  O programa principal deverá ler os lados e usar as funções para
  determinar se os lados realmente formam um triângulo e, em caso
  positivo, classificar o triângulo.

\paragraph{Dicas:}
\begin{itemize}
  \item A primeira função terá três argumentos (as medidas dos lados do
  triângulo) e um resultado (valor lógico indicando se os lados forma ou
  não um triângulo).
  \item A segunda funão também terá três argumentos e um resultado
  \emph{string} indicando a classificação do triângulo.
\end{itemize}

  \begin{runexample}
Triângulos
===============================
primeiro lado: 10
segundo lado: 10
terceiro lado: 10

classificação: equilátero
  \end{runexample}

  \begin{runexample}
Triângulos
===============================
primeiro lado: 10
segundo lado: 13
terceiro lado: 10

classificação: isósceles
   \end{runexample}

  \begin{runexample}
Triângulos
===============================
primeiro lado: 10
segundo lado: 8
terceiro lado: 15

classificação: escaleno
  \end{runexample}

  \begin{runexample}
Triângulos
===============================
primeiro lado: 3
segundo lado: 2
terceiro lado: 1

não é triângulo
  \end{runexample}

  \tcblower
  \solution
  \lstinput{scilab}{listings/p12/triangulo.sce}
\end{task}

\end{document}

