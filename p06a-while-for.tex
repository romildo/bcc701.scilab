\documentclass[11pt]{practice}

% Comente a linha seguinte para incluir as soluções
\tcbset{lowerbox=ignored}

\usetikzlibrary{calc}

\begin{document}

\institution{UFOP\quad DECOM}
\course{Programação de Computadores I}
\subtitle{Aula prática 6a}
\title{Comandos de Repetição: While --- Parte 2}
\author{}
\date{2013--2}
\maketitle

\begin{abstract}
  Nesta aula você desenvolverá algumas aplicações para treinar o uso do
  comando \texttt{while}.
\end{abstract}

\tableofcontents


\begin{task}[breakable]{Cálculo da média aritmética}{taf:media}
  O professor de Estatística deseja calcular a média aritmética das
  notas finais de seus alunos. Faça um programa que leia as notas dos
  alunos e calcula e exibe a sua média aritmética. A entrada dos dados
  deve ser encerrada quando for digitada uma nota negativa.

  \textbf{Dicas:}
  \begin{itemize}
    \item Leia a primeira nota antes de iniciar uma estrutura de
    repetição.
    \item Utilize o valor lido para controlar as repetições. As demais
    notas serão lidas no corpo da estrutura de repetição.
    \item Use uma variável para contar as notas válidas e outra variável
    para calcular a soma das notas válidas. O valor inicial destas
    variáveis deve ser zero (que é o elemento neutro da adição).
    \item Sempre que ler uma nota válida, atualize estas variáveis
    incrementando o contador em uma unidade, e adicionando a nota lida
    ao valor até então calculado para a soma.
    \item Ao terminar a entrada dos dados, verifique se a quantidade de
    notas válidas é positiva. Em caso afirmativo obtenha a média
    dividindo a soma pela quantidade de notas válidas.
  \end{itemize}

  \begin{runexample}
Cálculo da média
----------------------------
Digite a nota (valor negativo para terminar): 4.5
Digite a nota (valor negativo para terminar): 8
Digite a nota (valor negativo para terminar): 9.2
Digite a nota (valor negativo para terminar): 6.0
Digite a nota (valor negativo para terminar): 5.5
Digite a nota (valor negativo para terminar): 7.1
Digite a nota (valor negativo para terminar): -1

Média aritmética: 6.72
  \end{runexample}

  \begin{runexample}
Cálculo da média
----------------------------
Digite a nota (valor negativo para terminar): -5

Nenhuma nota válida foi digitada!
  \end{runexample}

  \tcblower
  \solution
  \lstinput{scilab}{listings/p06a/media.sce}
\end{task}

\begin{task}[breakable]{Cálculo do maior valor}{}
  O professor de Estatística também deseja determinar qual foi a maior
  nota no conjunto de notas de seus alunos (veja a tarefa anterior).

  Modife o programa da tarefa anterior para que seja também exibida a
  maior nota válida digitada pelo usuário.

  \textbf{Dicas:}
  \begin{itemize}
    \item Use uma variável para guardar a maior nota conhecida em um
    determinado momento. O valor inicial desta variável deve ser
    \texttt{-\%inf}, já que nenhuma nota será menor que $-\inf$.
    \item Sempre que ler uma nota válida, verifique se ela é maior do
    que a maior nota válida já encontrada até o momento. Em caso
    afirmativo armazene a nota em questão na variável, pois esta nota
    passa a ser a maior nota.
    \item Ao final do laço de repetição, caso tenham sido digitadas
    notas válidas, o valor desta variável será a maior nota válida.
  \end{itemize}

  \begin{runexample}
 Cálculo da média
----------------------------
Digite a nota (valor negativo para terminar): 6.4
Digite a nota (valor negativo para terminar): 8.9
Digite a nota (valor negativo para terminar): 1.5
Digite a nota (valor negativo para terminar): 4.3
Digite a nota (valor negativo para terminar): 9.2
Digite a nota (valor negativo para terminar): 6.0
Digite a nota (valor negativo para terminar): 5.9
Digite a nota (valor negativo para terminar): -1

Média aritmética: 6.03
Maior nota: 9.20
  \end{runexample}

  \begin{runexample}
Cálculo da média
----------------------------
Digite a nota (valor negativo para terminar): -5

Nenhuma nota válida foi digitada!
  \end{runexample}

  \tcblower
  \solution
  \lstinput{scilab}{listings/p06a/media2.sce}
\end{task}

\end{document}
