\documentclass[11pt,fleqn]{practice}
\usepackage{siunitx}

% Comente a linha seguinte para incluir as soluções
\tcbset{lowerbox=ignored}

\begin{document}

\institution{UFOP\quad DECOM}
\course{Programação de Computadores I}
\subtitle{Aula prática 4}
\title{Comandos de Desvio 2}
\author{}
\date{2014--2}
\maketitle

\begin{abstract}
  Nesta aula você irá resolver mais problemas que requerem uma decisão
  com base em um teste, ou condição. Para implementar a solução de
  problemas desse tipo, são utlizados comandos de desvio do fluxo de
  execução do programa (\textbf{if-then-else}).
\end{abstract}

\tableofcontents

\section{Condições mutuamente exclusivas}

Em algumas situações, podemos querer testar um conjunto de condições
mutuamente exclusivas, tal como na computação da função definida a
seguir:
\[ g(x,y) = \left\{\begin{array}{ll}
		       x + y     & \text{se}\; x \geq 0 \text{ e } y \geq 0 \\
		       x + y^2   & \text{se}\; x \geq 0 \text{ e } y < 0 \\
		       x ^2+ y   & \text{se}\; x < 0 \text{ e } y \geq 0 \\
		       x^2 + y^2 & \text{se}\; x < 0 \text{ e } y < 0               
                   \end{array} \right.
\]

Para computar o valor de $g(x,y)$ podemos escrever o seguinte trecho de
código em Scilab:
\begin{lst}{scilab}
if x >= 0 & y >= 0  then 
    gx = x + y                 // bloco 1
elseif x >= 0 & y < 0 then
    gx = x + y^2               // bloco 2
elseif x >= 0 & y < 0 then
    gx = x^2 + y               // bloco 3
else  // x < 0 & y < 0
    gx = x^2 + y^2             // bloco 4
end
\end{lst}

No comando acima, a condição da cláusula \texttt{if} é avaliada e, caso
seja verdadeira, o bloco de comnados 1 é executado. Caso contrário, é
avaliada a condição da cláusula \texttt{elseif} e, caso esta seja
verdadeira, o bloco 2 é executado, e assim por diante. Caso nenhuma das
condições anteriores seja verdadeira, é executado o bloco de comandos 4,
correspondente à cláusula \texttt{else}.

\section{Problemas}

\begin{task}[breakable]{Tarifa de Energia}{}
  A conta de energia elétrica de consumidores residenciais de uma cidade
  é calculada do seguinte modo, onde o consumo é dado em kilowatts (kw):
  \begin{itemize}
    \item Se o consumo é de até 500 kw, a tarifa é de R\$ 0,02 por
    unidade.
    \item Se o consumo é maior que 500 kw, mas não excede 1000 kw, a
    tarifa é de R\$ 0,10 para os 500 primeiros kw e de R\$ 0,05 para
    cada kw excedente a 500.
    \item Se o consumo é maior que 1000 kw, a tarifa é de R\$ 0,35 para
    os 1000 primeiros kw e de R\$ 0,10 para cada kw excedente a 1000.
    \item Em toda conta, é cobrada uma taxa básica de serviço de R\$
    5,00, independentemente da quantidade de energia consumida.
  \end{itemize}

  Escreva um programa que leia o consumo de energia elétrica de uma
  residência e imprima a sua conta de energia, no formato indicado no
  exemplo abaixo. O programa deve verificar se o valor fornecido para o
  consumo de energia é um valor inteiro positivo e, caso contrário,
  terminar exibindo uma mensagem indicativa de valor inválido.

  \begin{runexample}
Cálculo da Conta de Energia Elétrica
------------------------------------
Informe o consume de energia: 532.6
O consumo deve ser inteiro e positivo!
  \end{runexample}

  \begin{runexample}
Cálculo da Conta de Energia Elétrica
------------------------------------
Informe o consume de energia: 308
--------------------------------------
Consumo                    = 308 
Custo da energia consumida = R$   6.16 
Tarifa básica de serviço   = R$   5.00 
Valor a pagar              = R$  11.16 
--------------------------------------
  \end{runexample}

  \begin{runexample}
Cálculo da Conta de Energia Elétrica
------------------------------------
Informe o consume de energia: 547
--------------------------------------
Consumo                    = 547 
Custo da energia consumida = R$  52.35 
Tarifa básica de serviço   = R$   5.00 
Valor a pagar              = R$  57.35 
--------------------------------------
  \end{runexample}

  \begin{runexample}
Cálculo da Conta de Energia Elétrica
------------------------------------
Informe o consume de energia: 1123
--------------------------------------
Consumo                    = 1123 
Custo da energia consumida = R$ 362.30 
Tarifa básica de serviço   = R$   5.00 
Valor a pagar              = R$ 367.30 
--------------------------------------
  \end{runexample}

  \tcblower
  \solution
  \lstinput{scilab}{listings/p04/conta-energia.sce}
\end{task}

\begin{task}[breakable]{Aposentadoria}{}
  Um deputado propôs um projeto para alterar as regras para a
  aposentadoria. Por este projeto, para requerer a aposentadoria, os
  trabalhadores têm que combinar dois requisitos: tempo de contribuição
  ao INSS e idade mínima.

  Os trabalhadores do sexo masculino poderão aposentar-se com no mínimo
  50 anos de idade e no mínimo 30 anos de contribuição. Além disto, é
  necessário que a soma entre o tempo de contribuição e a idade seja de
  no mínimo 90 anos para eles.

  Faça um programa em Scilab que leia a idade e o tempo de contribuição
  de um trabalhador do sexo masculino e informe se o mesmo pode se
  aposentar. Não é necessário validar a idade e o tempo de contribuição.

  \begin{runexample}
Aposentadoria
------------------------------------------
Informe a idade em anos: 53
Informe o tempo de contribuição em anos: 35.6
Ainda não pode se aposentar.
  \end{runexample}

  \begin{runexample}
Aposentadoria
------------------------------------------
Informe a idade em anos: 54
Informe o tempo de contribuição em anos: 37
A aposentadoria pode ser solicitada.
  \end{runexample}

  \tcblower
  \solution
  \lstinput{scilab}{listings/p04/aposentadoria.sce}
\end{task}

\begin{task}[breakable]{IPTU}{}
  A prefeitura de Ouro Preto contratou você para fazer um programa que
  calcule os valores do IPTU de imóveis da cidade, conforme o tipo do
  loteamento e a área dos mesmos. Deverão ser considerados apenas dois
  tipos de loteamento: 1 e 2. Para cada tipo de loteamento, se a área do
  imóvel for menor que \SI{200}{m^2}, efetua-se um cálculo de IPTU; se
  for maior ou igual a \SI{200}{m^2}, efetua-se outro cálculo de IPTU. A
  tabela abaixo mostra como o cálculo deve ser efetuado para cada caso.

  \begin{center}
    \begin{tabular}{lll}\hline
      \textbf{tipo de loteamento} & $0 < \text{área} < \SI{200}{m^2}$ & $\text{área} \ge \SI{200}{m^2}$ \\\hline
      1 & $\text{IPTU} = \text{área} \times 1,0$ & $\text{IPTU} = \text{área} \times 1,2$ \\
      2 & $\text{IPTU} = \text{área} \times 1,1$ & $\text{IPTU} = \text{área} \times 1,3$ \\
      \hline
    \end{tabular}
  \end{center}

  Faça um programa em Scilab que leia o tipo de um loteamento e a área
  do mesmo e apresente o valor do IPTU de um determinado imóvel de Ouro
  Preto, calculado conforme a tabela acima.

  \begin{runexample}
Cálculo do IPTU
------------------------------------------
Informe o tipo do loteamento (1 ou 2): 1
Informe a área do imóvel em m^2: 150
O valor do IPTU é R$ 150.
  \end{runexample}

  \begin{runexample}
Cálculo do IPTU
------------------------------------------
Informe o tipo do loteamento (1 ou 2): 1
Informe a área do imóvel em m^2: 300
O valor do IPTU é R$ 360.
  \end{runexample}

  \begin{runexample}
Cálculo do IPTU
------------------------------------------
Informe o tipo do loteamento (1 ou 2): 2
Informe a área do imóvel em m^2: 300
O valor do IPTU é R$ 390.
  \end{runexample}

  \tcblower
  \solution
  \lstinput{scilab}{listings/p04/iptu.sce}
\end{task}

\begin{task}[breakable]{Linha de crédito}{}
  A prefeitura de Ouro Preto abriu uma linha de crédito para os
  funcionários celetistas. Qualquer funcionário pode solicitar um
  empréstimo, desde que o valor da prestação não ultrapasse 30\% de seu
  salário líquido. O salário líquido é obtido subtraindo-se do salário
  bruto a contribuição ao INSS (9\% do salário bruto).

  Codifique um programa que solicite ao usuário o valor do salário bruto
  e o valor da prestação que se deseja pagar. O programa deve informar
  se o empréstimo pode ou não ser concedido.

  \begin{runexample}
Linha de crédito
------------------------------------------
Digite o valor do salário bruto: 2519.65
Qual o valor da prestação a ser paga? 350.00
O empréstimo pode ser concedido!
Salário líquido: 2292.88
Valor máximo da prestação: 687.864
  \end{runexample}

  \begin{runexample}
Linha de crédito
------------------------------------------
Digite o valor do salário bruto: 1563.18
Qual o valor da prestação a ser paga? 427.00
O empréstimo não pode ser concedido!
Salário líquido: 1422.49
Valor máximo da prestação: 426.748
  \end{runexample}

  \tcblower
  \solution
  \lstinput{scilab}{listings/p04/credito.sce}
\end{task}

\end{document}
