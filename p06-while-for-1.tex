\documentclass[11pt,fleqn]{practice}

% Comente a linha seguinte para incluir as soluções
\tcbset{lowerbox=ignored}

\begin{document}

\institution{UFOP\quad DECOM}
\course{Programação de Computadores I}
\subtitle{Aula prática 6}
\title{Comandos de repetição --- \textbf{while}}
\author{}
\date{2014--2}
\maketitle

\begin{abstract}
Nesta aula vamos trabalhar com problemas cuja solução envolve realizar 
um cálculo ou tarefa repetidas vezes, enquanto uma determinada
condição é satisfeita. Em outras palavras, a implementação da solução
de tais problemas requer o uso de um comando de repetição, tal como o
comando \textbf{while}.
\end{abstract}

\tableofcontents

\section{Comandos de repetição - \texttt{while}}

A solução de diversos problemas, em computação, envolve a
\emph{repetição\/} de uma sequência de tarefas, ou comandos, enquanto
uma determinada \emph{condição\/} é satisfeita. Esse processo de
repetição, ou \emph{loop}, é implementado por meio do comando
\textbf{while}, que tem a seguinte sintaxe:

\begin{tabbing} xxxxxxxxxxxxxxxx\=ifxx\= \+\kill
      \textbf{while}  \emph{condição}  \textbf{do}\\
	   \>\emph{bloco de comandos  while} \\
     \textbf{end}
\end{tabbing}

A \emph{condição} deve ser uma expressão \emph{booleana}, isto é, cujo 
valor é verdadeiro (\texttt{\%t}) ou é falso (\texttt{\%f}). O bloco
de comandos é qualquer sequência de comandos, incluindo,
possivelmente, comandos de atribuição, de entrada e saída, de desvio
ou outros comandos de repetição.

A execução de um comando \textbf{while} é feita do seguinte modo: 
\begin{enumerate}
\item a \emph{condição\/} do \texttt{while} é avaliada; 
\item se a condição avalia como \texttt{\%t} (verdadeira), o
\emph{bloco de comandos while\/} é executado, e volta-se ao passo 1;
\item caso contrário, isto é, se a \emph{condição\/} avalia para
\texttt{\%f} (falso), comando \texttt{while} termina. A execução do
programa prossegue a partir do comando imediatamente subsequente ao
\texttt{end} do comando while.
\end{enumerate}

Note que, se a condição for inicialmente falsa, isto é, se o resultado 
for \texttt{\%f} na primeira vez em que a condição é avaliada, o bloco 
de comandos while não é executado nenhuma vez. Por outro lado, se o
valor da condição permanece sempre verdadeiro, em cada iteração do
comando while, a execução desse comando prossegue indefinidamente.

\section{Exemplos}

\subsection{Exibindo uma sequência de números naturais}

Escrever uma aplicação para exibir a sequência dos números naturais
menores que 25, usando a estrutura de repetição while.

\begin{runexample}
Contagem dos número naturais até 25:
1 2 3 4 5 6 7 8 9 10 11 12 13 14 15 16 17 18 19 20 21 22 23 24 25
\end{runexample}

O comando para exibir o número deverá ser escrito uma única vez no
programa, dentro de um comando de repetição, e ele será repetido várias
vezes. Em cada repetição ele exibirá um número diferente. Isto só será
possível se usarmos uma variável para armazenar o número a ser exibido,
e a cada repetição o valor da variável for modificado.

Assim para resolvermos este problema será usada uma variável para fazer
a contagem dos números naturais de 0 até 25. Chamaremos esta variável de
\texttt{contador}. O valor desta variável será exibido dentro do comando
de repetição.
\begin{itemize}
  \item O valor inicial da variável deverá ser 1, o primeiro valor a ser
  exibido.
  \item A cada repetição o valor da variável deverá ser incrementado em
  uma unidade, ou seja, \texttt{contador} receberá um novo valor igual
  ao seu successor do seu valor atual.
\end{itemize}
O corpo do comando de repetição deverá ser executado enquando a variável
\texttt{contador} estiver dentro do intervalo desejado. Logo a teste a
ser usado no comando while deve verificar se \texttt{contador} é menor
ou igual a 25.

Com estas considerações, chegamos ao seguinte programa:
\lstinput{scilab}{listings/p06/contagem.sce}


\pagebreak
\section{Resolvendo problemas}

\begin{task}[breakable]{Contagem dos números pares}{}
  Escreva um programa para exibir os números naturais pares dentro de um
  intervalo. O usuário deverá informar os limites do intervalo.

  Não é necessário fazer a validação da entrada: assumiremos que o
  usuário é inteligente o suficiente para digitar valores corretos.

  \begin{runexample}
ontagem dos número naturais pares
==================================
limite inferior do intervalo: 10
limite superior do intervalo: 25
10 12 14 16 18 20 22 24 
==================================
  \end{runexample}

  \tcblower
  \solution
  \lstinput{scilab}{listings/p06/contagempares.sce}
\end{task}

\newpage
\begin{task}[breakable]{Cálculo da média aritmética}{}
  Escreva um programa para calcular a média aritmética das notas obtidas
  pelos alunos de \emph{Programação de Computadores}. O usuário deverá
  primeiramente informar a quantidade de alunos da turma, e em seguida a
  nota obtida por cada aluno. Então o programa deve calcular e exibir a
  média aritmética das notas, caso o número de alunos seja positivo.

  \begin{runexample}
Cálculo da média aritmética das notas
=====================================

Informe a quantidade de alunos na turma: 0


Não há alunos na turma
  \end{runexample}

  \begin{runexample}
Cálculo da média aritmética das notas
=====================================

Informe a quantidade de alunos na turma: 5

Informe a nota do aluno: 9.4
Informe a nota do aluno: 5
Informe a nota do aluno: 6.5
Informe a nota do aluno: 7.5
Informe a nota do aluno: 2.6

Média calculada: 6.20
  \end{runexample}

  \textbf{Dicas:}\newline
  \begin{itemize}
    \item Use uma variável para armazenar o número total de alunos da
    turma.
    \item Use uma variável para calcular a soma das notas dos alunos. O
    seu valor inicial deve ser 0 (o elemento neutro da adição).
    \item Use uma variável para fazer a contagem dos alunos. O seu valor
    inicial deverá ser 1. Esta variável será usada para controlar as
    repetições.
    \item No corpo do comando de repetição:
    \begin{itemize}
      \item leia a nota de um aluno, armazenando-a em uma variável,
      \item atualize a variável da soma para acrescentar o nota lida à
      soma já calculada, e
      \item atualize a variavel contadora, para indicar que mais uma
      nota foi processada.
    \end{itemize}
    \item Após a finalização da entrada das notas, verifique se a
    quantidade de alunos realmente é positiva. Neste caso calcule e
    exiba a média. Caso contrário exiba uma mensagem adequada.
  \end{itemize}

  \tcblower
  \solution
  \lstinput{scilab}{listings/p06/mediaaritmetica.sce}
\end{task}

\newpage
\begin{task}[breakable]{Portaria da festa Baranga 2014}{}
  No ginásio da UFOP ocorrerá a festa Baranga 2014/2. O ingresso
  masculino será de R\$ 12,50 e o feminino será de R\$ 7,40.  Um calouro
  ficou encarregado de operar um programa na portaria do ginário para
  controlar o acesso das pessoas à festa. O programa é executado da
  seguinte forma:
  \begin{enumerate}
    \item Quando chega um homem na festa, ele digita \texttt{h}.
    \item Quando chega uma mulher na festa ele digita \texttt{m}.
    \item Quando o calouro quiser encerrar a entrada de dados ele digita
    \texttt{q}.
  \end{enumerate}
  O calouro não tem noção de quantas pessoas irão à festa.

  No momento que a entrada de dados for encerrada, o programa calcula
  quanto foi arrecadado com os ingressos masculinos e com os ingressos
  femininos. Também é calculado o total arrecadado.

  Codifique o programa operado pelo calouro na linguagem Scilab.

  \begin{runexample}
Portaria da festa BARANGA 2014/2
--------------------------------
Quem chegou (h/m/q): w
Quem chegou (h/m/q): i
Quem chegou (h/m/q): p
Quem chegou (h/m/q): q

Quantidade de homens ......: 0
Quantidade de mulheres ....: 0
Arrecadação com homens ....: R$ 0.00
Arrecadação com mulheres ..: R$ 0.00
Arrecadação total .........: R$ 0.00 
  \end{runexample}

  \begin{runexample}
Portaria da festa BARANGA 2014/2
--------------------------------
Quem chegou (h/m/q): m
Quem chegou (h/m/q): m
Quem chegou (h/m/q): h
Quem chegou (h/m/q): m
Quem chegou (h/m/q): h
Quem chegou (h/m/q): m
Quem chegou (h/m/q): h
Quem chegou (h/m/q): m
Quem chegou (h/m/q): m
Quem chegou (h/m/q): h
Quem chegou (h/m/q): q

Quantidade de homens ......: 4
Quantidade de mulheres ....: 6
Arrecadação com homens ....: R$ 50.00
Arrecadação com mulheres ..: R$ 44.40
Arrecadação total .........: R$ 94.40
  \end{runexample}

  \textbf{Dicas:}\newline
  \begin{itemize}
    \item A entrada de dados deve ser textual. Logo lembre-se de usar um
    segundo argumento na chamada da função \texttt{input} para indicar
    que será lida uma stringa.
    \item Utilize uma variável para contar os homens, e outra para
    contar as mulheres.
    \item Incremente a variável adequada quando o usuário digitar
    \texttt{h} ou \texttt{m}.
    \item O comando de repetição deve encerrar quando o usuário digitar
    \texttt{q}.
    \item A primeira entrada de dados deve ser feita antes de começar a
    repetir, pois a condição de repetição depende do valor digitado pelo
    usuário.
    \item As demais entradas de dados devem ser feitas no corpo do
    comando de repetição.
  \end{itemize}
  
  \tcblower
  \solution
  \lstinput{scilab}{listings/p06/Portaria.sce}
\end{task}

\end{document}
