\documentclass[11pt]{practice}

% Comente a linha seguinte para incluir as soluções
\tcbset{lowerbox=ignored}


\begin{document}

\institution{UFOP\quad DECOM}
\course{Programação de Computadores I}
\subtitle{Aula prática 13}
\title{Revisão 3a}
\author{}
\date{2013--2}
\maketitle

\begin{abstract}
  Esta prática é uma preparação para a terceira prova. Vamos lidar com
  problemas cuja solução envolve o uso des conceitos vistos até agora,
  destacando: vetores, matrizes e funções.
\end{abstract}

%\tableofcontents

\section{Exercícios}

\begin{task}[breakable]{Máximo divisor comum\footnote{Questão 4 da prova 3 de 2013/1}}{}

  O \textbf{algoritmo de Euclides} é utilizado para calcular o máximo
  divisor comum (MDC) de dois números inteiros. Abaixo é apresentado um
  trecho de programa que implementa este algoritmo usando um comando de
  repetição. Neste códido \texttt{divisor} e \texttt{dividendo} os
  números inteiros dos quais se deseja obter o MDC.
 
  \begin{minted}{scilab}
while divisor <> 0
   resto = modulo(dividendo, divisor)
   dividendo = divisor
   divisor = resto
end
  \end{minted}

  O algoritmo termina quando \texttt{divisor} é zero, e o MDC é o valor
  da variável \texttt{dividendo}.
 
  Dados dois números inteiros não nulos $n_1$ e $n_2$, o mínimo múltiplo
  comum (MMC) de $n_1$ e $n_2$ é calculado usando a seguinte equação (a
  qual usa o algoritmo de Euclides para o MDC):

  \[ \text{MMC}(n_1,n_2) = \frac{n_1 n_2}{\text{MDC}(n_1,n_2)} \]

  Escreva um programa principal e a função MDC em Scilab, onde:
  \begin{itemize}
    \item No programa principal:
    \begin{itemize}
      \item Faz-se a leitura de $n_1$ e $n_2$, dois números
      inteiros. Não é necessária a validação de dados; os números serão
      positivos e inteiros.
      \item O programa principal chama uma função definida pelo usuário
      \texttt{MDC}, com os argumentos $n_1$ e $n_2$, a qual retorna o
      máximo divisor comum desses números. Com o retorno da função, o
      programa principal calcula o MMC de $n_1$ e $n_2$, imprimindo o
      resultado.
      \item Caso $n_1$ ou $n_2$, ou ambos, sejam nulos, o MMC não pode
      ser calculado; neste caso é exibida uma mensagem de aviso ao
      usuário e o programa é encerrado.
    \end{itemize}
    \item Na função definida pelo usuário \texttt{MDC}:
    \begin{itemize}
      \item Utiliza-se o trecho de programa descrito acima.
    \end{itemize}
  \end{itemize}

  \begin{runexample}
Cálculo do mínimo múltiplo comum de dois números
=========================================================
Digite um inteiro........: 10
Digite outro inteiro.....: 0
O mínimo múltiplo comum não pode ser calculado.
  \end{runexample}

  \begin{runexample}
Cálculo do mínimo múltiplo comum de dois números
=========================================================
Digite um inteiro........: 544
Digite outro inteiro.....: 119
MMC(544, 119) = 3808
  \end{runexample}

  \tcblower
  \solution
  \lstinput{scilab}{listings/p13/mdc.sce}
\end{task}


\begin{task}[breakable]{Temperaturas máximas\footnote{Questão 1 da prova 3 de 2013/1}}{}

  As temperaturas máximas diárias (em $^o$F) para Chicago e São
  Francisco durante o mês de agosto de 2009 são dadas nos vetores abaixo
  (dados da Administração Nacional Oceânica e Atmosférica dos EUA).

  \begin{minted}{scilab}
TCH = [75 79 86 86 79 81 73 89 91 86 81 82 86 88 89 ...
       90 82 84 81 79 73 69 73 79 82 72 66 71 69 66 66];

TSF = [69 79 70 73 72 71 69 76 85 87 74 84 76 68 79 ...
       75 68 68 81 72 79 68 68 69 71 70 89 95 90 66 69];
  \end{minted}

  Escreva um programa para determinar quantos dias, e em que datas, no
  mês dado, a temperatura foi a mesma nas duas cidades.

  \textbf{Observações:}\\
  \begin{itemize}
    \item Exemplificando, no quinto dia de agosto a temperatura em
    Chicago foi de 79$^o$F, e em São Francisco foi de 72$^o$F.
    \item Os vetores \texttt{TCH} e \texttt{TSF} já estão definidos. O
    usuário não precisa fazer nenhuma entrada de dados.
  \end{itemize}

  \begin{runexample}
Datas em que ocorreram a mesma temperatura:  2 19 30
Quantidade de dias que ocorreram a mesma temperatura: 3
  \end{runexample}

  \tcblower
  \solution
  \lstinput{scilab}{listings/p13/temperaturas.sce}
\end{task}

\end{document}
