\documentclass[11pt]{practice}

% Comente a linha seguinte para incluir as soluções
\tcbset{lowerbox=ignored}

\begin{document}

\institution{UFOP\quad DECOM}
\course{Programação de Computadores I}
\subtitle{Aula prática 12}
\title{Vetores: Terceira Parte}
\author{}
\date{2014--1}
\maketitle

\begin{abstract}
  As atividades propostas nesta prática visam explorar o uso de vetores
  para o desenvolvimento de aplicações.
\end{abstract}

\tableofcontents

\section{Exercícios}

\begin{task}[breakable]{}{}
  Codificar um programa Scilab que leia um vetor de $n$ valores. A
  seguir, o programa determina e exibe quantos elementos são nulos,
  positivos e negativos.
  \begin{runexample}
Digite a quantidade de elementos do vetor:5
Início da leitura dos elementos do vetor...
Elemento 1: 
  digite o valor --> 10
Elemento 2: 
  digite o valor --> 20
Elemento 3: 
  digite o valor --> 30
Elemento 4: 
  digite o valor --> 0
Elemento 5: 
  digite o valor --> -13

Vetor original:
10  20  30  0  -13  

Elementos nulos     --> 1
Elementos positivos --> 3
Elementos negativos --> 1
  \end{runexample}

  \tcblower
  \solution
  \lstinput{scilab}{listings/p12/pos-neg-nul.sce}
\end{task}

\begin{task}[breakable]{}{}
  Uma escola deseja saber se existem alunos cursando, simultaneamente,
  as disciplinas Programação de Computadores e Cálculo Numérico. Coloque
  os números das matrículas doa alunos que cursam Programação de
  Computadores em um vetor, e dos alunos que cursam Cálculo Numérico em
  outro vetor. Mostre o número das matrículas que aparecem nos dois
  vetores simultaneamente.

  \begin{runexample}
Alunos matriculados em Programação de Computadores:
matrícula (0 para terminar): 10
matrícula (0 para terminar): 20
matrícula (0 para terminar): 30
matrícula (0 para terminar): 44
matrícula (0 para terminar): 50
matrícula (0 para terminar): 55
matrícula (0 para terminar): 60
matrícula (0 para terminar): 70
matrícula (0 para terminar): 0

Alunos matriculados em Programação de Computadores:
matrícula (0 para terminar): 5
matrícula (0 para terminar): 15
matrícula (0 para terminar): 20
matrícula (0 para terminar): 33
matrícula (0 para terminar): 40
matrícula (0 para terminar): 50
matrícula (0 para terminar): 55
matrícula (0 para terminar): 80
matrícula (0 para terminar): 90
matrícula (0 para terminar): 0

Alunos matriculados nos nois cursos:
20 50 55
  \end{runexample}

  \tcblower
  \solution
  \lstinput{scilab}{listings/p12/matriculas.sce}
\end{task}

\end{document}
