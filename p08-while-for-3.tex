\documentclass[11pt,fleqn]{practice}

% Comente a linha seguinte para incluir as soluções
\tcbset{lowerbox=ignored}

\begin{document}

\institution{UFOP\quad DECOM}
\course{Programação de Computadores I}
\subtitle{Aula prática 7}
\title{Comandos de repetição --- \textbf{for}}
\author{}
\date{2014--2}
\maketitle

\begin{abstract}
  Nesta aula você desenvolverá algumas aplicações para treinar o uso do
  comando \texttt{for}.
\end{abstract}

\tableofcontents

\begin{task}[breakable]{Valor de uma série}{}
  Desenvolva um programa Scilab para calcular o valor aproximado da série
  \[ \frac{1}{2} + \frac{1}{4} + \frac{1}{6} + \frac{1}{8} + \ldots +
  \frac{1}{2*i} + \ldots \]

  O programa deverá solicitar ao usuário quantos termos devem ser usados
  no cálculo, e a seguir deve calcular e exibir o valor da série.

  \begin{runexample}
Cálculo da série
=========================
Digite a quantidade de termos: 4
Valor da série com 4 parcelas: 1.04167
  \end{runexample}
  \tcblower
  \solution
  \lstinput{scilab}{listings/p08/serie1.sce}
\end{task}

\begin{task}[breakable]{Valor de uma série}{}
  Faça um programa para calcular o valor de S, dado por
  \[ S = \frac{1}{n} + \frac{2}{n-1} + \frac{3}{n-2} + \ldots + \frac{n-1}{2} +
  \frac{n}{1} \]
  sendo $n$ fornecido pelo usuário através do teclado.

  \begin{runexample}
Cálculo da série:
-----------------------------
Digite a quantidade de parcelas: 100
Valor da série com 100 termos: 423.925
  \end{runexample}

  \tcblower
  \solution
  \lstinput{scilab}{listings/p08/serie2.sce}
\end{task}

\begin{task}[breakable]{Valor aproximado de $\pi$}{}
  Uma aproximação para o valor de $\pi$ pode ser calculada pela seguinte
  fórmula:

  \[ \pi = \sqrt{12} ( 1 - \frac{1}{3\times 3} + \frac{1}{5\times 3^2} - \frac{1}{7\times 3^3} + \frac{1}{9\times 3^4} - \cdots ) \]

  Quanto maior for o número de termos usados na soma da série, mais
  próximo do valor de $\pi$ é o valor calculado. Isso é ilustrado na
  tabela exibida no exemplo de execução do programa, onde a primeira
  coluna mostra o número de termos usados no cálculo, a segunda coluna
  mostra o valor obtido usando esse número de termos e a terceira coluna
  mostra o erro absoluto entre o valor real (valor predefinido do
  Scilab) e valor calculado, isto é, módulo da diferença desses valores.

  Escreva um programa que leia um valor inteiro positivo $n$ ,
  representando o número de parcelas a serem incluídas no
  somatório e imprima a tabela de execução do programa ilustrada abaixo.

  Observação: Não utilizar vetores na implementação.

  \begin{runexample}
Cálculo aproximado de pi
-----------------------------------
Valor de pi pré-definido no scilab: 3.1415926536
Digite o número de termos da série: 10
-----------------------------------------
termos  pi calculado  erro absoluto
-----------------------------------------
     1  3.4641016151  0.32250896155
     2  3.0792014357  0.06239121791
     3  3.1561814716  0.01458881798
     4  3.1378528916  0.00373976199
     5  3.1426047457  0.00101209207
     6  3.1413087855  0.00028386813
     7  3.1416743127  0.00008165911
     8  3.1415687159  0.00002393765
     9  3.1415997738  0.00000712022
    10  3.1415905109  0.00000214265
  \end{runexample}

  \textbf{Dicas:} Para calcular cada termo da série:
  \begin{itemize}
    \item use uma variável para representar o sinal do termo. O valor
    incial desta variável deverá ser 1, e ela deverá ser invertida
    (multiplicada por -1) a cada passo do laço de repetição para que
    alterne entre 1 e -1.
    \item use uma variável para representar o primeiro fator do
    denominador. O valor incial desta variável deverá ser 1, e ela
    deverá ser incrementada em duas unidades a cada passo do laço de
    repetição.
    \item use uma variável para representar expoente do segundo fator do
    denominador. O valor incial desta variável deverá ser 0, e ela
    deverá ser incrementada em uma unidade a cada passo do laço de
    repetição.
  \end{itemize}
  \tcblower
  \solution
  \lstinput{scilab}{listings/p08/pi.sce}
\end{task}

\begin{task}[breakable]{Função de duas variáveis}{}
  Seja a função de duas variáveis:

  \[ 
  f(x,y) =
    \begin{cases}
      x^2 - 3x + y^2  & \text{se $x < y$}\\
      \frac{\sqrt{y^2 - 4x}}{2}  & \text{se $x = y$}\\
      \sqrt[3]{xy}  & \text{se $x < y$}
    \end{cases}
  \]

  Escreva um programa para gerar a tabela de valores dessa função
  (conforme o exemplo a seguir), para valores de $x$ e $y$ nos seguintes
  intervalos:
  \begin{itemize}
    \item $0 \leq x \leq 1$ (com incrementos de 0,1 em $x$)
    \item $0 \leq y \leq 1,4$ (com incrementos de 0,2 em $y$)
  \end{itemize}

  \begin{runexample}
Tabela da função de duas variáveis
----------------------------------
x/y |    0.0    0.2    0.4    0.6    0.8    1.0    1.2    1.4
-------------------------------------------------------------
0.0 |   0.00   0.04   0.16   0.36   0.64   1.00   1.44   1.96
0.1 |   0.00  -0.25  -0.13   0.07   0.35   0.71   1.15   1.67
0.2 |   0.00   0.00  -0.40  -0.20   0.08   0.44   0.88   1.40
0.3 |   0.00   0.39  -0.65  -0.45  -0.17   0.19   0.63   1.15
0.4 |   0.00   0.43   0.00  -0.68  -0.40  -0.04   0.40   0.92
0.5 |   0.00   0.46   0.58  -0.89  -0.61  -0.25   0.19   0.71
0.6 |   0.00   0.49   0.62   0.00  -0.80  -0.44   0.00   0.52
0.7 |   0.00   0.52   0.65   0.75  -0.97  -0.61  -0.17   0.35
0.8 |   0.00   0.54   0.68   0.78   0.00  -0.76  -0.32   0.20
0.9 |   0.00   0.56   0.71   0.81   0.90  -0.89  -0.45   0.07
1.0 |   0.00   0.58   0.74   0.84   0.93   0.00  -0.56  -0.04
  \end{runexample}

  \textbf{Observação:} Não precisa exibir a primeira linha e a primeira
  coluna da tabela.

  \tcblower
  \solution
  \lstinput{scilab}{listings/p08/tabela.sce}
\end{task}

\end{document}
