\documentclass[11pt]{practice}

% Comente a linha seguinte para incluir as soluções
\tcbset{lowerbox=ignored}

\begin{document}

\institution{UFOP\quad DECOM}
\course{Programação de Computadores I}
\subtitle{Aula prática 14}
\title{Revisão 3b}
\author{}
\date{2013--2}
\maketitle

\begin{abstract}
  Esta prática é uma preparação para a terceira prova. Vamos lidar com
  problemas cuja solução envolve o uso des conceitos vistos até agora,
  destacando: vetores, matrizes e funções.
\end{abstract}

%\tableofcontents

\section{Exercícios}

\begin{task}[breakable]{Vendas de uma loja\footnote{Questão 2 da prova 3 de 2013/1}}{}

  Escreva um programa que, a partir de uma matriz de vendas semanais dos
  vendedores de uma loja, calcule e imprima:
  \begin{itemize}
    \item o total de vendas de cada semana (todos os vendedores somados)
    \item o total de vendas de cada vendedor (todas as semanas somadas)
    \item o total de vendas (todos os valores somados)
  \end{itemize}

  Você não precisa ler os valores das vendas semanais: suponha que eles
  já estão armazenados em uma matriz de nome $A$, onde cada linha
  representa um vendedor e cada coluna representa uma semana. Você deve
  imprimir essa matriz (podendo usar o comando \texttt{disp}). Os
  números de semanas e de vendedores não são conhecidos antecipadamente,
  e devem ser determinados pelas dimensões da matriz $A$.

  O exemplo de execução a seguir ilustra um caso em que a matriz $A$
  corresponde às vendas de 5 vendedores, durante 4 semanas.

  \begin{runexample}
Matriz de vendas: 
    10.    31.    28.    11.  
    37.    42.    33.    11.  
    0.     34.    36.    10.  
    16.    43.    9.     44.  
    33.    3.     27.    32.  

Vendas da semana 1: 96
Vendas da semana 2: 153
Vendas da semana 3: 133
Vendas da semana 4: 108

Vendas do vendedor 1: 80
Vendas do vendedor 2: 123
Vendas do vendedor 3: 80
Vendas do vendedor 4: 112
Vendas do vendedor 5: 95

Vendas no mês: 490
  \end{runexample}

  \tcblower
  \solution
  \lstinput{scilab}{listings/p14/vendas.sce}
\end{task}

\end{document}
