\documentclass[11pt]{practice}

% Comente a linha seguinte para incluir as soluções
\tcbset{lowerbox=ignored}

\begin{document}

\institution{UFOP\quad DECOM}
\course{Programação de Computadores I}
\subtitle{Aula prática 12}
\title{Vetores: Primeira Parte}
\author{}
\date{2014--2}
\maketitle

\begin{abstract}
  As atividades propostas nesta prática visam explorar as primeiras
  noções sobre vetores para o desenvolvimento de aplicações.
\end{abstract}

\tableofcontents

\section{Vetores e matrizes}

A unidade fundamental de dados em Scilab é a \textbf{matriz}. Uma matriz
é semelhante a uma tabela, exceto que uma matriz pode ter qualquer
número de dimensões, enquanto uma tabela geralmente tem apenas duas
dimensões, que usualmente são chamadas de linhas e colunas.

Por exemplo, as matrizes $A$, $B$ e $C$, a seguir, têm dimensões
$1\times 5$, $3\times 1$ e $3\times 2$, respectivamente:
\[
\begin{array}{lll}
  A = \begin{array}{|r|r|r|r|r|} \hline
        3 & 5 & 7 & 12 & 9 \\ \hline
      \end{array}
  &
  \hspace*{1cm}
  B = \begin{array}{|r|} \hline
        8 \\ \hline
        5 \\ \hline
        7 \\ \hline
      \end{array}
  &
  \hspace*{1cm}
  C = \begin{array}{|r|r|} \hline
        3 & 8 \\ \hline
        12 & 5 \\ \hline
        9 & 7 \\ \hline
      \end{array} 
\end{array}
\]
A matriz $A$, que tem apenas 1 linha, é também chamada de
\textbf{vetor-linha}, ou simplesmente \textbf{vetor}.  A matriz $B$, que
tem apenas 1 coluna, é também chamada de \textbf{vetor-coluna}.

Em Scilab, o valor denotado por uma variável é sempre uma matriz. Em
particular, um valor escalar, tal como $2$, $11.3$ ou \texttt{\%pi} é
visto como uma matriz de dimensão $1\times 1$.

\subsection{Criando Matrizes}

Uma matriz pode ser especificada simplesmente enumerando os seus
elementos. Em Scilab, os elementos de uma matriz devem ser especificados
entre colchetes; elementos de uma mesma linha são separados por espaço
ou por vírgula (\texttt{,}), e as linhas são separadas por
mudança-de-linha ou por ponto-e-vírgula (\texttt{;}).

Os exemplos a seguir ilustram os comandos Scilab para criação das 3
matrizes \texttt{A}, \texttt{B} e \texttt{C} mostradas
anteriormente. Digite cada um desses comandos no {\emph prompt} do
console do Scilab e observe o resultado. Note também, na janela de
variáveis, as dimensões das variáveis \texttt{A}, \texttt{B} e
\texttt{C} criadas por estes comandos.

\begin{lst}{text}
-->A = [ 3  2*8-5  -7 \%pi ]
 A =
 
     3.    11.    -7.    3.1415927  

-->B = [0.5; 2; 10.3] 
 B  =
 
      0.5
      2.
      10.3
 
-->C = [ 10, 2, 5; 3, 2.3, 0.6]
 C  =
 
     10.    2.      5.
     3.      2.3    0.6
     
-->D = [ 10, 2, 5; 3, 7.2]
 D  =
                       !-- error 6
Inconsistent row/column dimensions
\end{lst} 

\subsubsection*{Observações}
\begin{itemize}
  \item O valor de um elemento da matriz pode ser especificado como uma
  expressão arbitrária.
  \item Cada linha da matriz deve ter o mesmo número de elementos, e
  cada coluna deve ter o mesmo número de linhas.
\end{itemize}  

\subsection{Indexando matrizes}

Uma posição em uma matriz \texttt{M} de dimensão $n\times m$ é
identificada por um par de \emph{índices} $(i,j)$, onde $i$ é o número
(ou índice) da linha e $j$ é o número (ou índice) da coluna desta
posição na matriz. Por exemplo, \texttt{M(2,3)} especifica o elemnto na
linha 2, coluna 3 da matriz \texttt{M}. Esse mecanismo de
\emph{indexação} é ilustrado a seguir.

\begin{lst}{text}

--> A = [10, 20, 30]
 A  =
 
     10.     20.      30.

---> x = A(1,3)        
 x = 

     30.

\end{lst}

A indexação de uma matriz deve ser feita de modo que os valores dos
índices estejam compreeendidos dentro dos limtes da dimensão dessa
matriz. Isto é, se \texttt{M} é uma matriz de dimensão $m\times n$,
\texttt{M(i,j)} deve ser tal que $1 \leq \texttt{i} \leq n$ e
$1 \leq \texttt{j} \leq m$. Caso contrário, é reportado um erro de
índice inválido, como ilustra o exemplo a seguir.

\begin{lst}{text}
-->z = A(2,3)           // erro de indexação: A tem apenas 1 linha
               ! error 21
Invalid index
\end{lst}

A indexação em um vetor-linha pode ser feita omitindo-se o número da
linha e especificando apenas o número da coluna do elemento desejado,
conforme ilustrado a seguir. De maneira similar, a indexação em um
vetor-coluna pode ser feita omitindo-se o número da coluna e
especificando apenas o número da linha do elemento desejado.

\begin{lst}{text}
--> A = [10, 20, 30]
 A  =
 
     10.     20.      30.

---> y = A(2)          
 x = 

     20. 


---> A(3) = 45      
A  =
 
     10.     20.      45.
\end{lst}

\subsubsection*{Observações}
\begin{itemize}
  \item Um índice de linha ou coluna em uma matriz deve ser um valor
  inteiro positivo, e pode ser especificado como uma expressão
  arbitrária.
\end{itemize}  

\subsection{Obtendo as dimensões de uma matriz}

As seguintes funções podem ser usadas, em Scilab, para determinar o
número de elementos ou as dimensões de uma matriz.

\begin{center}
  \begin{tabular}{lp{8cm}} \hline
    \textbf{função}                & \textbf{descrição}                \\\hline
    \texttt{$n$ = length($A$)}     & número de elementos de $A$        \\\hline
    \texttt{[$l$,$c$] = size($A$)} & número de linhas e colunas de $A$ \\\hline
  \end{tabular}
\end{center}

Exemplo:
\begin{lst}{scilab}
-->A = [ 1 8 9 5; 2 3 4 0 ]
 A  =
 
    1.    8.    9.    5.  
    2.    3.    4.    0.  

-->dim = size(A)
 dim  =
 
      2.    4.  
 
-->[lin,col] = size(A)
 col  =
 
    4.  
 lin  =
 
    2.  

-->length(A)
 ans  =

     8.

-->length(A(1,:))
 ans  =
 
      4.

-->length(A(:,2))
ans  =
 
      2.
\end{lst}

\subsubsection*{Observações}
\begin{itemize}
  \item O resultado da expressão \texttt{(size(M)}, onde \texttt{M} é
  uma matriz de $n$ dimensões, ($n\geq 2$) consiste de um vetor de $n$
  valores.
  \item O resultado de uma expressão \texttt{(size(M)} pode ser
  atribuído a uma variável, tal como na atribuição \texttt{dim =
    size(C)} acima.
  \item O resultado de uma expressão \texttt{(size(M)}, onde \texttt{M}
  é uma matriz de $n$ dimensões, pode também ser atribuído a um vetor de
  $n$ variáveis \texttt{[$d_1$, $d_2$, \ldots, $d_n$]}, sendo que cada
  variável $d_i$, para $1 \leq i \leq n$, irá conter o número de
  elementos da dimensão $i$ da matriz (tal como no último exemplo
  acima).
\end{itemize}

\newpage

\section{Exemplo de aplicação}

Escrever um programa em Scilab para ler uma sequência de notas digitadas
pelo usuário, e calcular e exibir a média aritmética das notas, bem como
a quantidade de notas que estão acima desta média. A entrada deve ser
encerrada quando o usuário digitar um valor negativo.

\begin{runexample}
Cálculo da média
---------------------------
digite a nota: 6.8
digite a nota: 9.0
digite a nota: 8.1
digite a nota: 4.5
digite a nota: 7.0
digite a nota: 1.9
digite a nota: 6.0
digite a nota: 10.0
digite a nota: 6.8
digite a nota: 7.5
digite a nota: -1

média: 6.76
7 notas estao acima da média
\end{runexample}

\lstinput{scilab}{listings/p12/acima-media.sce}

\newpage

\section{Exercícios}

\begin{task}{Entrada e saída de vetores}{}

  Faça um programa que leia um vetor com cinco posições para números
  reais e, depois, um código inteiro. Se o código for zero, finalize o
  programa; se for 1, mostre o vetor na ordem direta; se for 2, mostre o
  vetor na ordem inversa.

  \begin{runexample}
digite um elemento: 2
digite um elemento: 4
digite um elemento: 0
digite um elemento: 8
digite um elemento: 5

digite o código (0, 1 ou 2): 2

elementos do vetor em ordem inversa:
5 8 0 4 2
  \end{runexample}

  \tcblower
  \solution
  \lstinput{scilab}{listings/p12/ler-e-exibir-vetor.sce}
\end{task}

\begin{task}{Cálculo com elementos de vetores}{}
  Codifique um programa que preencha dois vetores de seis elementos
  através de entradas pelo teclado. Após a definição dos dois vetores,
  construa um terceiro vetor onde cada elemento é igual ao dobro da soma
  entre os elementos correspondentes dos outros dois vetores. Imprima o
  conteúdo do vetor calculado.

  \begin{runexample}
digite os elementos do primeiro vetor:
digite um elemento: 1
digite um elemento: 2
digite um elemento: 3
digite um elemento: 4

digite os elementos do segundo vetor:
digite um elemento: 5
digite um elemento: 6
digite um elemento: 7
digite um elemento: 8

elementos do vetor calculado:
12 16 20 24
  \end{runexample}

  \tcblower
  \solution
  \lstinput{scilab}{listings/p12/operacao-com-vetor.sce}
\end{task}

\begin{task}{Contagem de números negativos, nulos e positivos}{}
  Codificar um programa Scilab que leia um vetor de $n$ valores. A
  seguir, o programa determina e exibe quantos elementos são nulos,
  positivos e negativos.

  A quantidade $n$ de elementos no vetor é determinada pelo usuário.
  \begin{runexample}
Digite a quantidade de elementos do vetor:5
Início da leitura dos elementos do vetor...
Elemento 1: 
  digite o valor --> 10
Elemento 2: 
  digite o valor --> 20
Elemento 3: 
  digite o valor --> 30
Elemento 4: 
  digite o valor --> 0
Elemento 5: 
  digite o valor --> -13

Vetor original:
10  20  30  0  -13  

Elementos nulos     --> 1
Elementos positivos --> 3
Elementos negativos --> 1
  \end{runexample}

  \tcblower
  \solution
  \lstinput{scilab}{listings/p12/pos-neg-nul.sce}
\end{task}

\begin{task}{Temperaturas máximas}{}

  As temperaturas máximas diárias (em $^o$F) para Chicago e São
  Francisco durante o mês de agosto de 2009 são dadas nos vetores abaixo
  (dados da Administração Nacional Oceânica e Atmosférica dos EUA).

  \begin{minted}{scilab}
TCH = [75 79 86 86 79 81 73 89 91 86 81 82 86 88 89 ...
       90 82 84 81 79 73 69 73 79 82 72 66 71 69 66 66];

TSF = [69 79 70 73 72 71 69 76 85 87 74 84 76 68 79 ...
       75 68 68 81 72 79 68 68 69 71 70 89 95 90 66 69];
  \end{minted}

  Escreva um programa para determinar quantos dias, e em que datas, no
  mês dado, a temperatura foi a mesma nas duas cidades.

  \textbf{Observações:}\\
  \begin{itemize}
    \item Exemplificando, no quinto dia de agosto a temperatura em
    Chicago foi de 79$^o$F, e em São Francisco foi de 72$^o$F.
    \item Os vetores \texttt{TCH} e \texttt{TSF} já estão definidos. O
    usuário não precisa fazer nenhuma entrada de dados.
  \end{itemize}

  \begin{runexample}
Datas em que ocorreram a mesma temperatura:  2 19 30
Quantidade de dias que ocorreram a mesma temperatura: 3
  \end{runexample}

  \tcblower
  \solution
  \lstinput{scilab}{listings/p12/temperaturas.sce}
\end{task}

% \begin{task}{Pesquisa de notas de alunos}{}
%   Nesta tarefa você desenvolverá uma aplicação para pesquisar a nota de
%   um aluno.

%   Codifique um programa que defina por atribuição os vetores
%   \texttt{aluno} e \texttt{notasBCC701} abaixo:

% \begin{verbatim}
% aluno = [ "Peny" "Rajesh Koothrappali" "Sheldon Cooper" ...
%           "Howard Wolowitz" "Leonard Hofstadter" ]

% notasBCC701 = [ 6.0 8.5 10.0 9.0 9.5 ]
% \end{verbatim}

%   Observe que os vetores possuem um relacionamento através de suas
%   posições correspondentes. Assim, o nome que estiver armazenado no
%   vetor \texttt{aluno}, na posição \texttt{k}, possui a nota armazenada
%   na posição correspondente \texttt{k} do vetor \texttt{notasBCC701}.

%   O programa solicita ao usuário um determinado nome, e a seguir,
%   imprime sua respectiva nota na disciplina BCC701.

%   \begin{runexample}
% Informe o nome do aluno: Howard Wolowitz
% Nota em BCC701: 9.0
%   \end{runexample}

%   \tcblower
%   \solution
%   \lstinput{scilab}{listings/p12/pesquisa-nota.sce}
% \end{task}


\end{document}
