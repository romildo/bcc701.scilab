\documentclass[11pt]{practice}

% Comente a linha seguinte para incluir as soluções
\tcbset{lowerbox=ignored}

\begin{document}

\institution{UFOP\quad DECOM}
\course{Programação de Computadores I}
\subtitle{Aula prática 10}
\title{Vetores: Segunda Parte}
\author{}
\date{2014--1}
\maketitle

\begin{abstract}
  As atividades propostas nesta prática visam explorar o uso de vetores
  para o desenvolvimento de aplicações.
\end{abstract}

\tableofcontents

\section{Exercícios}

\begin{task}[breakable]{Soma cumulativa de um vetor}{}
  Codifique um programa que preencha um vetor com entradas feitas pelo
  usuário através do teclado. Considere que o usuário definirá apenas
  valores numéricos não negativos, e que, ao desejar encerrar a entrada
  dos elementos do vetor ele digite um valor negativo.

  Após a entrada de todos os elementos do vetor, o programa gera outro
  vetor, o qual representa o resultado da execução da função
  \texttt{cumsum} pré-definida no Scilab que calcula a soma cumulativa
  de um vetor (porém não use a função \texttt{cumsum}). O elemento na
  posição $i$ do vetor soma cumulativa de $v$ é dado pela soma de todos
  os elementos de $v$ com índices de $1$ a $i$, inclusive. Por exemplo,
  a soma cumulativa do vetor
  \[ (1, 8, 6, 10) \]
  é o vetor
  \[ (1, 1+8, 1+8+6, 1+8+6+10) \]
  que é igual a
  \[ (1, 9, 15, 25) \]

  \begin{runexample}
Cálculo da soma acumulada de um vetor
-------------------------------------
Digite os elementos do vetor.
Para encerrar digite um número negativo
  digite um elemento do vetor: -5

Vetor dado:
[ ]

Soma cumulativa do vetor dado:
[ ]
  \end{runexample}

  \begin{runexample}
Cálculo da soma acumulada de um vetor
-------------------------------------
Digite os elementos do vetor.
Para encerrar digite um número negativo
  digite um elemento do vetor: 1
  digite um elemento do vetor: 8
  digite um elemento do vetor: 6
  digite um elemento do vetor: 10
  digite um elemento do vetor: -2

Vetor dado:
[ 1 8 6 10 ]

Soma cumulativa do vetor dado:
[ 1 9 15 25 ]
  \end{runexample}

  \tcblower
  \solution
  \lstinput{scilab}{listings/p11/soma-cumulativa.sce}
\end{task}


\end{document}
