\documentclass[11pt,fleqn]{practice}

% Comente a linha seguinte para incluir as soluções
\tcbset{lowerbox=ignored}

\begin{document}

\institution{UFOP\quad DECOM}
\course{Programação de Computadores I}
\subtitle{Aula prática 7}
\title{Comandos de repetição --- \textbf{while}}
\author{}
\date{2014--2}
\maketitle

\begin{abstract}
  Nesta aula você desenvolverá algumas aplicações para treinar o uso do
  comando \texttt{while}.
\end{abstract}

\tableofcontents

\section{Resolvendo problemas}

\begin{task}[breakable]{Sequência de Collatz}{}
  A conjectura de Collatz, também conhecida como conjectura $3n+1$, foi
  proposta pelo matemático Lothar Collatz, em 1937. Para explicar essa
  conjectura, considere o seguinte processo que descreve como obter a
  sequência de Collatz para um número inteiro $n>0$:

  \begin{quote}
    Se $n$ for par, divida $n$ por 2, obtendo $n/2$; se $n$ for ímpar,
    multiplique $n$ por $3$ e some $1$, obtendo $3n+1$. Repita esse
    processo para o valor obtido, e assim sucessivamente, até que o
    valor obtido seja $1$.
  \end{quote}

  A tabela seguinte mostra algumas sequências de Collatz.
  \begin{center}
    \begin{tabular}{|r|l|} \hline
      $\mathbf{n}$ & \textbf{sequência de Collatz para $\mathbf{n}$} \\\hline
      5 & 5, 16, 8, 4, 2, 1 \\\hline
      11 & 11, 34, 17, 52, 26, 13, 40, 20, 10, 5, 16, 8, 4, 2, 1 \\\hline
      12 & 12, 6, 3, 10, 5, 16, 8, 4, 2, 1 \\\hline
    \end{tabular}
  \end{center}

  A conjectura é que esse processo de cálculo sempre termina: sempre se
  obtém, eventualmente, o valor 1, para qualquer inteiro $n>0$ dado
  inicialmente. Tal conjectura nunca foi provada, mas também nunca se
  encontrou um exemplo em contrário.

  Escreva um programa que leia um valor inteiro $n>0$ e imprima a
  sequência de Collatz para $n$, assim como o comprimento dessa
  sequência. O programa deve verificar se o valor de entrada é válido,
  solicitando um novo valor caso não o seja.

  \begin{runexample}
Sequência de Collatz
---------------------
Digite um número inteiro não negativo: 11

Sequência: 11 34 17 52 26 13 40 20 10 5 16 8 4 2 1
Comprimento da sequência: 15
  \end{runexample}

  \begin{runexample}
Sequência de Collatz
---------------------
Digite um número inteiro não negativo: -8
Valor inválido.
Digite um número inteiro não negativo: -2
Valor inválido.
Digite um número inteiro não negativo: 0
Valor inválido.
Digite um número inteiro não negativo: 5

Sequência: 5 16 8 4 2 1
Comprimento da sequência: 6
  \end{runexample}

  \tcblower
  \solution
  \lstinput{scilab}{listings/p07/Collatz.sce}
\end{task}

\newpage
\begin{task}[breakable]{Valor aproximado de $\pi$}{}
  O valor de $\pi$ pode ser aproximado pela seguinte série:

  \[ \frac{\pi}{8} = \frac{1}{1 \times 3} +  \frac{1}{5 \times 7} +  \frac{1}{9 \times 11} + \cdots \]

  Escreva um programa que leia um valor $n \ge 1$ e calcule um valor
  aproximado para $\pi$ usando $n$ termos da série acima.
  
  O seu programa deve fazer a validação da entrada. Caso o número
  digitado pelo usuário seja inválido, deve-se exibir uma função
  apropriada e solicitar ao usuário que digite novamente.

  \begin{runexample}
Cálculo do valor aproximado de pi
-----------------------------------
Quantidade de iterações: -5
  Valor inválido. Tente novamente.

Quantidade de iterações: 0
  Valor inválido. Tente novamente.

Quantidade de iterações: 5

pi = 3.041839618929
  \end{runexample}

  \begin{runexample}
Cálculo do valor aproximado de pi
-----------------------------------
Quantidade de iterações: 10

pi = 3.091623806668
  \end{runexample}

  \begin{runexample}
Cálculo do valor aproximado de pi
-----------------------------------
quantidade de iterações: 100

pi = 3.136592684839
  \end{runexample}

  \tcblower
  \solution
  \lstinput{scilab}{listings/p07/pi-aprox1.sce}
\end{task}

\newpage
\begin{task}[breakable]{Aplicações financeiras}{}
  Em uma empresa trabalham dois amigos, Charlie e Alan. Eles são
  ambiciosos e desejam economizar para futuramente abrirem sua própria
  empresa. Todo mês Charlie aplicará 80\% do seu salário na caderna de
  poupança, que está rendendo 2\% ao mês, e Alan aplicará 50\% do seu
  salário no fundo de renda fixa, que está rendendo 5\% ao
  mês.

  Escreva um programa que receba os salários de Charlie e de Alan, e
  calcula e mostra a quantidade de meses necessários para que o valor da
  aplicação de pertencente a Alan ultrapasse o valor da aplicação
  pertencente a Charlie.

  \begin{runexample}
Comparação de duas aplicações
======================================
Informe o salário de Charlie ...: 3000
Informe o salário de Alan ......: 2000
Após 49 meses de aplicação:
Rendimentos de Charlie: 196657.42
Rendimentos de Alan: 198426.66
  \end{runexample}

  \tcblower
  \solution
  \lstinput{scilab}{listings/p07/aplicacoes.sce}
\end{task}


\end{document}
