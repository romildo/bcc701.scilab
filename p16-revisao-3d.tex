\documentclass[11pt]{practice}

% Comente a linha seguinte para incluir as soluções
\tcbset{lowerbox=ignored}

\begin{document}

\institution{UFOP\quad DECOM}
\course{Programação de Computadores I}
\subtitle{Aula prática 16}
\title{Revisão 3d}
\author{}
\date{2013--2}
\maketitle

\begin{abstract}
  Esta prática é uma preparação para a terceira prova. Vamos lidar com
  problemas cuja solução envolve o uso des conceitos vistos até agora,
  destacando: vetores, matrizes e funções.
\end{abstract}

%\tableofcontents

\section{Exercícios}

\begin{task}[breakable]{Conversão de temperaturas}{}

  O programa listado a seguir, quando executado, exibe uma tabela de
  correspondência de temperaturas nas escalas Celsius e Fahrenheit, de
  $32^oC$ até $44^oC$, com incrementos de $2^oC$.

  \begin{lst}{scilab}
clc; clear;
printf("Tabela de temperaturas\n")
printf("-------------------------------\n");
printf("T(C)    T(F)\n");
for t = 0 : 2 : 30
    printf("%4.0f   %5.1f\n", t, fahrenheit(t));
end
  \end{lst}

  Observe que o programa usa uma função chamada \texttt{fahrenheit},
  cuja definição foi omitida.

  Defina a função \texttt{fahrenheit} para converter uma temperatura da
  escala Celsius para a escala Fahrenheit segundo a equação
  \[ C = \frac{9}{5} F + 32 \] onde $C$ é a temperatura na escala
  Celsius e $F$ é a temperatura na escala Fahrenheit.

  \begin{runexample}
Tabela de temperaturas
-------------------------------
T(C)    T(F)
   0    32.0
   2    35.6
   4    39.2
   6    42.8
   8    46.4
  10    50.0
  12    53.6
  14    57.2
  16    60.8
  18    64.4
  20    68.0
  22    71.6
  24    75.2
  26    78.8
  28    82.4
  30    86.0
  \end{runexample}

  \tcblower
  \solution
  \lstinput{scilab}{listings/p16/temperaturas.sce}
\end{task}

\end{document}
